%! SKRÓTY KLAWISZOWE
% LINK do skrótów klawiszowych: https://github.com/James-Yu/latex-workshop/wiki/Snippets
% ctrl+alt+j - przeniesienie z kodu do pdf
% ctrl + click - przeniesienie z pdf do kodu (dokument.pdf)
% zaznaczony fragment kodu -> ctrl+l -> ctrl+w
% gdy kuror na sekcji itp. -> cltr + alt + ] - obniżenie sekcji
% gdy kuror na sekcji itp. -> cltr + alt + [ - podniesienie sekcji
% kopia lini kodu -> ctrl + shift + strzałka w dół
\newpage
\section{Założenia projektowe – wymagania}		%1

% * SPOSOBY UZYWANIA MAKRA HERE * # 
% ? Listingi
% Parametr #1: Opis listingu (wyświetla sie bezpośrednio pod listingiem)
% Parametr #2 : Nazwa pliku oraz ID do oznaczania (wazne, zeby byl w katalogu kod oraz jego rozszerzenie to txt) 

\ListingFile{OpisPromptu}{prompt}

Tak wstawiamy listingi, za pomocą:

  \begin{verbatim}
	ListingFile\{Opis listingu}{nazwa-pliku} 
 \end{verbatim}
	
	wstawiamy listingi z plików z katalogu kod.

Gdzie:
\begin{verbatim}
	{Opis listingu} - opis listingu wyświetlany pod listingiem
	{nazwa-pliku} - nazwa pliku z katalogu kod i jednocześnie ID do oznaczania
\end{verbatim}

Tak oznaczamy listingi \OznaczKod{prompt} w tekście.\\Za pomocą:

\begin{verbatim}
	ListingFile\{Opis listingu}{nazwa-pliku} 
\end{verbatim}

gdzie:

\begin{verbatim}
	{nazwa-pliku} - nazwa pliku z katalogu kod i jednocześnie ID do oznaczania
\end{verbatim}



\clearpage
% ? Zdjęcia
% OPCJONALNY Parametr #1: Szerokość zdjęcia (domyślnie jest to szerokość paragrafu ale jak sie poda w [#1] to wtedy zmienia się na podaną wartość)
% Parametr #2: Nazwa pliku z rozszerzeniem (podajemy katalog w którym jest plik i jego rozszerzenie np. rys/nazwa-rysunku.png)
% Parametr #3: Opis tego co jest na rysunku (wyświetla sie bezpośrednio pod rysunkiem) 
% Parametr #4: Identyfikator rysunku (do oznaczania zdjęć w tekście) 

\fg{rys/nazwa-rysunku.png}{Opis tego co jest na rysunku}{id-rysunku}

jest równe 
\fg[\textwidth]{rys/nazwa-rysunku.png}{Opis tego co jest na rysunku}{id-rysunku}
\clearpage
Tak wstawiamy zdjęcia, za pomocą:

  \begin{verbatim}
	\fg{szerokość-rysunku}{nazwa-pliku}{Opis rysunku}{id-rysunku} 
 \end{verbatim}

 Gdzie:
\begin{verbatim}
	{szerokość-rysunku} - szerokość rysunku domyślnie \textwidth
	{nazwa-pliku} - nazwa pliku z katalogu rys i jego rozszerzenie
	{Opis rysunku} - opis rysunku wyświetlany pod rysunkiem
	{id-rysunku} - identyfikator rysunku
\end{verbatim}

 Tak oznaczamy zdjęcia \OznaczZdjecie{id-rysunku} w tekście. Za pomocą:

  \begin{verbatim}
	\OznaczZdjecie{id-rysunku}
	\end{verbatim}



