%╔════════════════════════════╗
%║	  Szablon dostosował	  ║
%║	mgr inż. Dawid Kotlarski  ║
%║		  06.10.2024		  ║
%╚════════════════════════════╝
\documentclass[12pt,twoside,a4paper,openany]{article}

    % ------------------------------------------------------------------------
% PAKIETY
% ------------------------------------------------------------------------

%różne pakiety matematyczne, warto przejrzeć dokumentację, muszą być powyżej ustawień językowych.
\usepackage{mathrsfs}   %Różne symbole matematyczne opisane w katalogu ~\doc\latex\comprehensive. Zamienia \mathcal{L} ze zwykłego L na L-transformatę.
\usepackage{eucal}      %Różne symbole matematyczne.
\usepackage{amssymb}    %Różne symbole matematyczne.
\usepackage{amsmath}    %Dodatkowe funkcje matematyczne, np. polecenie \dfac{}{} skladajace ulamek w trybie wystawionym (porównaj $\dfrac{1}{2}$, a $\frac{1}{2}$).

%język polski i klawiatura
\usepackage[polish]{babel}
\usepackage{csquotes}
%\usepackage{qtimes} % czcionka Times new Roman
\usepackage{polski}

\usepackage{ifluatex}

\ifluatex
  %czcionka
  \usepackage{fontspec}
  \setmainfont{Calibri}

  %obsługa pdf'a
  \usepackage[luatex,usenames,dvipsnames]{color}      %Obsługa kolorów. Opcje usenames i dvipsnames wprowadzają dodatkowe nazwy kolorow.
  \usepackage[luatex,pagebackref=false,draft=false,pdfpagelabels=false,colorlinks=true,urlcolor=cyan,linkcolor=blue,filecolor=magenta,citecolor=green,pdfstartview=FitH,pdfstartpage=1,pdfpagemode=UseOutlines,bookmarks=true,bookmarksopen=true,bookmarksopenlevel=2,bookmarksnumbered=true,pdfauthor={Dawid Kotlarski},pdftitle={Dokumentacja Projektowa},pdfsubject={},pdfkeywords={transient recovery voltage trv},unicode=true]{hyperref}   %Opcja pagebackref=true dotyczy bibliografii: pokazuje w spisie literatury numery stron, na których odwołano się do danej pozycji.
\else
  \usepackage[pdftex,usenames,dvipsnames]{color}      %Obsługa kolorów. Opcje usenames i dvipsnames wprowadzają dodatkowe nazwy kolorow.
  \usepackage[pdftex,pagebackref=false,draft=false,pdfpagelabels=false,colorlinks=true,urlcolor=cyan,linkcolor=blue,filecolor=magenta,citecolor=green,pdfstartview=FitH,pdfstartpage=1,pdfpagemode=UseOutlines,bookmarks=true,bookmarksopen=true,bookmarksopenlevel=2,bookmarksnumbered=true,pdfauthor={Dawid Kotlarski},pdftitle={Dokumentacja Projektowa},pdfsubject={},pdfkeywords={transient recovery voltage trv},unicode=true]{hyperref}   %Opcja pagebackref=true dotyczy bibliografii: pokazuje w spisie literatury numery stron, na których odwołano się do danej pozycji.
\fi

%bibliografia
%\usepackage[numbers,sort&compress]{natbib}  %Porządkuje zawartość odnośników do literatury, np. [2-4,6]. Musi być pod pdf'em, a styl bibliogfafii musi mieć nazwę z dodatkiem 'nat', np. \bibliographystyle{unsrtnat} (w kolejności cytowania).
\usepackage[
  backend=biber,
  style=numeric,
  sorting=none
]{biblatex}
\addbibresource{bibliografia.bib}
\usepackage{hypernat}                       %Potrzebna pakietowi natbib do wspolpracy z pakietem hyperref (wazna kolejnosc: 1. hyperref, 2. natbib, 3. hypernat).

%grafika i geometria strony
\usepackage{extsizes}           %Dostepne inne rozmiary czcionek, np. 14 w poleceniu: \documentclass[14pt]{article}.
\usepackage[final]{graphicx}
\usepackage[a4paper,left=3.5cm,right=2.5cm,top=2.5cm,bottom=2.5cm]{geometry}

%strona tytułowa
\usepackage{strona_tytulowa}

%inne
\usepackage{lastpage} %! do numerowania stron w formacie (x z y)
\usepackage[hide]{todo}                     %Wprowadza polecenie \todo{treść}. Opcje pakietu: hide/show. Polecenie \todos ma byc na koncu dokumentu, wszystkie \todo{} po \todos sa ignorowane.
\usepackage[basic,physics]{circ}            %Wprowadza środowisko circuit do rysowania obwodów elektrycznych. Musi byc poniżej pakietow językowych.
\usepackage[sf,bf,outermarks]{titlesec}     %Troszczy się o wygląd tytułów rozdziałów (section, subsection, ...). sf oznacza czcionkę sans serif (typu arial), bf -- bold. U mnie: oddzielna linia dla naglowku paragraph. Patrz tez: tocloft -- lepiej robi format spisu tresci.
\usepackage{tocloft}                        %Troszczy się o format spisu trsci.
\usepackage{expdlist}    %Zmienia definicję środowiska description, daje większe możliwości wpływu na wygląd listy.
\usepackage{flafter}     %Wprowadza parametr [tb] do polecenia \suppressfloats[t] (polecenie to powoduje nie umieszczanie rysunkow, tabel itp. na stronach, na ktorych jest to polecenie (np. moze byc to stroma z tytulem rozdzialu, ktory chcemy zeby byl u samej gory, a nie np. pod rysunkiem)).
\usepackage{array}       %Ładniej drukuje tabelki (np. daje wiecej miejsca w komorkach -- nie są tak ścieśnione, jak bez tego pakietu).
\usepackage{listings}    %Listingi programow.
\usepackage[format=hang,labelsep=period,labelfont={bf,small},textfont=small]{caption}   %Formatuje podpisy pod rysunkami i tabelami. Parametr 'hang' powoduje wcięcie kolejnych linii podpisu na szerokosc nazwy podpisu, np. 'Rysunek 1.'.
\usepackage{appendix}    %Troszczy się o załączniki.
\usepackage{floatflt}    %Troszczy się o oblewanie rysunkow tekstem.
\usepackage{here}        %Wprowadza dodtkowy parametr umiejscowienia rysunków, tabel, itp.: H (duże). Umiejscawia obiekty ruchome dokladnie tam gdzie są w kodzie źródłowym dokumentu.
\usepackage{makeidx}     %Troszczy się o indeks (skorowidz).

%nieużywane, ale potencjalnie przydatne
\usepackage{sectsty}           %Formatuje nagłówki, np. żeby były kolorowe -- polecenie: \allsectionsfont{\color{Blue}}.
%\usepackage{version}           %Wersje dokumentu.

%============
\usepackage{longtable}			%tabelka
%============

%============
% Ustawienia listingów do kodu
%============

\usepackage{listings}
\usepackage{xcolor}

\definecolor{codegreen}{rgb}{0,0.6,0}
\definecolor{codegray}{rgb}{0.5,0.5,0.5}
\definecolor{codepurple}{rgb}{0.58,0,0.82}
\definecolor{backcolour}{rgb}{0.95,0.95,0.92}

% Definicja stylu "mystyle"
\lstdefinestyle{mystyle}{
backgroundcolor=\color{backcolour},
commentstyle=\color{codegreen},
keywordstyle=\color{blue},	%magenta
numberstyle=\tiny\color{codegray},
stringstyle=\color{codepurple},
basicstyle=\ttfamily\footnotesize,
breakatwhitespace=false,
breaklines=true,
captionpos=b,
keepspaces=true,
numbers=left,
numbersep=5pt,
showspaces=false,
showstringspaces=false,
showtabs=false,
tabsize=2,
literate=
  {á}{{\'a}}1 {é}{{\'e}}1 {í}{{\'i}}1 {ó}{{\'o}}1 {ú}{{\'u}}1
{Á}{{\'A}}1 {É}{{\'E}}1 {Í}{{\'I}}1 {Ó}{{\'O}}1 {Ú}{{\'U}}1
{à}{{\`a}}1 {è}{{\`e}}1 {ì}{{\`i}}1 {ò}{{\`o}}1 {ù}{{\`u}}1
{À}{{\`A}}1 {È}{{\`E}}1 {Ì}{{\`I}}1 {Ò}{{\`O}}1 {Ù}{{\`U}}1
{ä}{{\"a}}1 {ë}{{\"e}}1 {ï}{{\"i}}1 {ö}{{\"o}}1 {ü}{{\"u}}1
{Ä}{{\"A}}1 {Ë}{{\"E}}1 {Ï}{{\"I}}1 {Ö}{{\"O}}1 {Ü}{{\"U}}1
{â}{{\^a}}1 {ê}{{\^e}}1 {î}{{\^i}}1 {ô}{{\^o}}1 {û}{{\^u}}1
{Â}{{\^A}}1 {Ê}{{\^E}}1 {Î}{{\^I}}1 {Ô}{{\^O}}1 {Û}{{\^U}}1
{ã}{{\~a}}1 {ẽ}{{\~e}}1 {ĩ}{{\~i}}1 {õ}{{\~o}}1 {ũ}{{\~u}}1
{Ã}{{\~A}}1 {Ẽ}{{\~E}}1 {Ĩ}{{\~I}}1 {Õ}{{\~O}}1 {Ũ}{{\~U}}1
{œ}{{\oe}}1 {Œ}{{\OE}}1 {æ}{{\ae}}1 {Æ}{{\AE}}1 {ß}{{\ss}}1
{ű}{{\H{u}}}1 {Ű}{{\H{U}}}1 {ő}{{\H{o}}}1 {Ő}{{\H{O}}}1
{ç}{{\c c}}1 {Ç}{{\c C}}1 {ø}{{\o}}1 {Ø}{{\O}}1 {å}{{\r a}}1 {Å}{{\r A}}1
{€}{{\euro}}1 {£}{{\pounds}}1 {«}{{\guillemotleft}}1
{»}{{\guillemotright}}1 {ñ}{{\~n}}1 {Ñ}{{\~N}}1 {¿}{{?`}}1 {¡}{{!`}}1
{ą}{{\k{a}}}1 {ć}{{\'{c}}}1 {ę}{{\k{e}}}1 {ł}{{\l}}1 {ń}{{\'n}}1
{ó}{{\'o}}1 {ś}{{\'s}}1 {ź}{{\'z}}1 {ż}{{\.{z}}}1
{Ą}{{\k{A}}}1 {Ć}{{\'{C}}}1 {Ę}{{\k{E}}}1 {Ł}{{\L}}1 {Ń}{{\'N}}1
{Ó}{{\'O}}1 {Ś}{{\'S}}1 {Ź}{{\'Z}}1 {Ż}{{\.{Z}}}1
}

\lstset{style=mystyle} % Deklaracja aktywnego stylu
%===========

%PAGINA GÓRNA I DOLNA
\usepackage{fancyhdr}          %Dodaje naglowki jakie się chce.
\pagestyle{fancy}
\fancyhf{}
% numery stron w paginie dolnej na srodku
\fancyfoot[C]{\footnotesize DOKUMENTACJA PROJEKTU – SYSTEMY OPERACYJNE  \\
  \normalsize\sffamily  \thepage\ z~\pageref{LastPage}}


%\fancyhead[L]{\small\sffamily \nouppercase{\leftmark}}
\fancyhead[C]{\footnotesize \textit{AKADEMIA NAUK STOSOWANYCH W NOWYM SĄCZU}\\}

\renewcommand{\headrulewidth}{0.4pt}
\renewcommand{\footrulewidth}{0.4pt}

    % ------------------------------------------------------------------------
% USTAWIENIA
% ------------------------------------------------------------------------

% ------------------------------------------------------------------------
%   Kropki po numerach sekcji, podsekcji, itd.
%   Np. 1.2. Tytuł podrozdziału
% ------------------------------------------------------------------------
\makeatletter
    \def\numberline#1{\hb@xt@\@tempdima{#1.\hfil}}                      %kropki w spisie treści
    \renewcommand*\@seccntformat[1]{\csname the#1\endcsname.\enspace}   %kropki w treści dokumentu
\makeatother

% ------------------------------------------------------------------------
%   Numeracja równań, rysunków i tabel
%   Np.: (1.2), gdzie:
%   1 - numer sekcji, 2 - numer równania, rysunku, tabeli
%   Uwaga ogólna: o otoczeniu figure ma być najpierw \caption{}, potem \label{}, inaczej odnośnik nie działa!
% ------------------------------------------------------------------------
\makeatletter
    \@addtoreset{equation}{section} %resetuje licznik po rozpoczęciu nowej sekcji
    \renewcommand{\theequation}{{\thesection}.\@arabic\c@equation} %dodaje kropki

    \@addtoreset{figure}{section}
    \renewcommand{\thefigure}{{\thesection}.\@arabic\c@figure}

    \@addtoreset{table}{section}
    \renewcommand{\thetable}{{\thesection}.\@arabic\c@table}
\makeatother

% ------------------------------------------------------------------------
% Tablica
% ------------------------------------------------------------------------
\newenvironment{tabela}[3]
{
    \begin{table}[!htb]
    \centering
    \caption[#1]{#2}
    \vskip 9pt
    #3
}{
    \end{table}
}

% ------------------------------------------------------------------------
% Dostosowanie wyglądu pozycji listy \todos, np. zamiast 'p.' jest 'str.'
% ------------------------------------------------------------------------
\renewcommand{\todoitem}[2]{%
    \item \label{todo:\thetodo}%
    \ifx#1\todomark%
        \else\textbf{#1 }%
    \fi%
    (str.~\pageref{todopage:\thetodo})\ #2}
\renewcommand{\todoname}{Do zrobienia...}
\renewcommand{\todomark}{~uzupełnić}

% ------------------------------------------------------------------------
% Definicje
% ------------------------------------------------------------------------
\def\nonumsection#1{%
    \section*{#1}%
    \addcontentsline{toc}{section}{#1}%
    }
\def\nonumsubsection#1{%
    \subsection*{#1}%
    \addcontentsline{toc}{subsection}{#1}%
    }
\reversemarginpar %umieszcza notki po lewej stronie, czyli tam gdzie jest więcej miejsca
\def\notka#1{%
    \marginpar{\footnotesize{#1}}%
    }
\def\mathcal#1{%
    \mathscr{#1}%
    }
\newcommand{\atp}{ATP/EMTP} % Inaczej: \def\atp{ATP/EMTP}

% ------------------------------------------------------------------------
% Inne
% ------------------------------------------------------------------------
\frenchspacing                      
\hyphenation{ATP/-EMTP}             %dzielenie wyrazu w danym miejscu
\setlength{\parskip}{3pt}           %odstęp pomiędzy akapitami
\linespread{1.3}                    %odstęp pomiędzy liniami (interlinia)
\setcounter{tocdepth}{4}            %uwzględnianie w spisie treści czterech poziomów sekcji
\setcounter{secnumdepth}{4}         %numerowanie do czwartego poziomu sekcji 
\titleformat{\paragraph}[hang]      %wygląd nagłówków
{\normalfont\sffamily\bfseries}{\theparagraph}{1em}{}

%komenda do łatwiejszego wstawiania zdjęć
\newcommand*{\fg}[4][\textwidth]{
    \begin{figure}[!htb]
        \begin{center}
            \includegraphics[width=#1]{#2}
            \caption{#3}
            \label{rys:#4}
        \end{center}
    \end{figure}
}

\newcommand*{\Oznacz}[2]{
\ref{#1:#2} (s. \pageref{#1:#2})
}

\newcommand*{\OznaczZdjecie}[2][Rysunek]{
#1 \Oznacz{rys}{#2}
}
    
\newcommand*{\OznaczKod}[1]{
\Oznacz{lst}{#1}
}

\newcommand*{\ListingFile}[2]{
    \lstinputlisting[caption=#1, label={lst:#2}, language=C++]{kod/#2.txt}
}


    %polecenia zdefiniowane w pakiecie strona_tytulowa.sty
    \title{Zaprojektować i wdrożyć system informatyczny na
    potrzeby przedsiębiorstwa zgodnie z założeniami}		%...Wpisać nazwę projektu...
    \author{Imie Nazwisko}
    \authorI{}
    \authorII{}		%jeśli są dwie osoby w projekcie to zostawiamy:    \authorII{}
		
	\uczelnia{AKADEMIA NAUK STOSOWANYCH \\W NOWYM SĄCZU}
    \instytut{Wydział Nauk Inżynieryjnych}
    \kierunek{Katedra Informatyki}
    \praca{DOKUMENTACJA PROJEKTOWA}
    \przedmiot{SYSTEMY OPERACYJNE}
    \prowadzacy{mgr inż. Jan Kozieński}
    \rok{2025}


%definicja składni mikrotik
\usepackage{fancyvrb}
\DefineVerbatimEnvironment{MT}{Verbatim}%
{commandchars=\+\[\],fontsize=\small,formatcom=\color{red},frame=lines,baselinestretch=1,} 
\let\mt\verb 
%zakonczenie definicji składni mikrotik

\usepackage{fancyhdr}    %biblioteka do nagłówka i stopki

			
\begin{document}
   
    \renewcommand{\figurename}{Rys.}    %musi byc pod \begin{document}, bo w~tym miejscu pakiet 'babel' narzuca swoje ustawienia
    \renewcommand{\tablename}{Tab.}     %j.w.
    \thispagestyle{empty}               %na tej stronie: brak numeru
    \stronatytulowa                     %strona tytułowa tworzona przez pakiet strona_tytulowa.tex
 
 \pagestyle{fancy}

    \newpage

    %formatowanie spisu treści i~nagłówków
    \renewcommand{\cftbeforesecskip}{8pt}
    \renewcommand{\cftsecafterpnum}{\vskip 8pt}
    \renewcommand{\cftparskip}{3pt}
    \renewcommand{\cfttoctitlefont}{\Large\bfseries\sffamily}
    \renewcommand{\cftsecfont}{\bfseries\sffamily}
    \renewcommand{\cftsubsecfont}{\sffamily}
    \renewcommand{\cftsubsubsecfont}{\sffamily}
    \renewcommand{\cftparafont}{\sffamily}
    %koniec formatowania spisu treści i nagłówków
     
    \tableofcontents    %spis treści
    \thispagestyle{fancy}
    \newpage

    
    \newpage

    
%%%%%%%%%%%%%%%%%%% treść główna dokumentu %%%%%%%%%%%%%%%%%%%%%%%%%

   %! Pobieranie PDF
% ctrl + shift + ~
% komenda: upload.sh

%! SKRÓTY KLAWISZOWE
% LINK do skrótów klawiszowych: https://github.com/James-Yu/latex-workshop/wiki/Snippets
% ctrl+alt+j - przeniesienie z kodu do pdf
% ctrl + click - przeniesienie z pdf do kodu (dokument.pdf)
% zaznaczony fragment kodu -> ctrl+l -> ctrl+w
% gdy kuror na sekcji itp. -> cltr + alt + ] - obniżenie sekcji
% gdy kuror na sekcji itp. -> cltr + alt + [ - podniesienie sekcji
% kopia lini kodu -> ctrl + shift + strzałka w dół
\newpage
\section*{Tutorial}		%1

% * SPOSOBY UZYWANIA MAKRA HERE * # 
% ? Listingi
% Parametr #1: Opis listingu (wyświetla sie bezpośrednio pod listingiem)
% Parametr #2 : Nazwa pliku oraz ID do oznaczania (wazne, zeby byl w katalogu kod oraz jego rozszerzenie to txt) 

\ListingFile{OpisPromptu}{prompt}

Tak wstawiamy listingi, za pomocą:

\begin{verbatim}
	ListingFile\{Opis listingu}{nazwa-pliku} 
 \end{verbatim}

wstawiamy listingi z plików z katalogu kod.

Gdzie:
\begin{verbatim}
	{Opis listingu} - opis listingu wyświetlany pod listingiem
	{nazwa-pliku} - nazwa pliku z katalogu kod i jednocześnie ID do oznaczania
\end{verbatim}

Tak oznaczamy listingi \OznaczKod{prompt} w tekście.\\Za pomocą:

\begin{verbatim}
	ListingFile\{Opis listingu}{nazwa-pliku} 
\end{verbatim}

gdzie:

\begin{verbatim}
	{nazwa-pliku} - nazwa pliku z katalogu kod i jednocześnie ID do oznaczania
\end{verbatim}

\clearpage
% ? Zdjęcia
% OPCJONALNY Parametr #1: Szerokość zdjęcia (domyślnie jest to szerokość paragrafu ale jak sie poda w [#1] to wtedy zmienia się na podaną wartość)
% Parametr #2: Nazwa pliku z rozszerzeniem (podajemy katalog w którym jest plik i jego rozszerzenie np. rys/nazwa-rysunku.png)
% Parametr #3: Opis tego co jest na rysunku (wyświetla sie bezpośrednio pod rysunkiem) 
% Parametr #4: Identyfikator rysunku (do oznaczania zdjęć w tekście) 

\fg{rys/nazwa-rysunku.png}{Opis tego co jest na rysunku}{id-rysunku}

jest równe
\fg[\textwidth]{rys/nazwa-rysunku.png}{Opis tego co jest na rysunku}{id-rysunku}
\clearpage
Tak wstawiamy zdjęcia, za pomocą:

\begin{verbatim}
	\fg{szerokość-rysunku}{nazwa-pliku}{Opis rysunku}{id-rysunku} 
 \end{verbatim}

Gdzie:
\begin{verbatim}
	{szerokość-rysunku} - szerokość rysunku domyślnie \textwidth
	{nazwa-pliku} - nazwa pliku z katalogu rys i jego rozszerzenie
	{Opis rysunku} - opis rysunku wyświetlany pod rysunkiem
	{id-rysunku} - identyfikator rysunku
\end{verbatim}

Tak oznaczamy zdjęcia \OznaczZdjecie{id-rysunku} w tekście. Za pomocą:

\begin{verbatim}
	\OznaczZdjecie{id-rysunku}
	\end{verbatim}



\section{Założenia projektowe – wymagania}		%1
\begin{enumerate}
    \item Utworzenie własnej domeny AD według formatu firma.ad, gdzie firma to nazwa firmy dla której
          przygotowujemy projekt.
    \item Autoryzacja pracownika przy użyciu imiennego konta działającego na wszystkich komputerach w
          sieci firmowej. Login zgodny ze schematem: imie.nazwisko. Utworzyć minimum po trzy konta dla
          każdego wydziału. Tworzenie grup oraz kont w domenie wykonywać przez skrypt, który odczyta
          dane grupy, konta z pliku i utworzy w domenie AD.
    \item Pracownicy powinni należeć do grupy globalnej odpowiedniej dla wydziału w którym
          pracują(przewidzieć 5 przykładowych wydziałów, np. kadry, place, gospodarczy, marketing, itp wg.
          własnego uznania). W przedsiębiorstwie przewidziano wydział informatyczny, którego pracownicy
          mają w pełni administrować domeną przedsiębiorstwa. Tworzenie grup oraz kont w domenie
          wykonywać przez skrypt, który odczyta dane grupy, konta z pliku i utworzy w domenie AD.
    \item Pracownicy powinni korzystać z zasobów sieciowych o nazwach: wspolny, oraz zasób wydziałowy
          (oddzielny zasób dla każdego wydziału). Zasoby powinny być udostępnione poprzez klaster pracy
          awaryjnej, który powinien korzystać z przestrzeni dyskowej (macierzy RAID-1) udostępnionej
          poprzez iSCSI o przestrzeni wypadkowej 30GB. Jeśli pozwalają zasoby sprzętowe to programową
          macierz RAID-1, oraz iSCSI Target Serwer można zainstalować na oddzielnym serwerze o nazwie
          SMPXX.firma.ad.
    \item Zasób wspolny ma być mapowany użytkownikowi jako dysk (patrz: założenia projektowej),
          natomiast zasób wydziałowy jako jeden z dysków (patrz: założenia projektowej) adekwatnie do
          grupy wydziałowej w której się pracownik znajduje. Mechanizm mapowania automatyczny przy
          użyciu polis GPO.
    \item System ma umożliwić instalację stacji klienckich z obrazu udostępnionego na serwerze.
    \item Pracownicy powinni mieć dostęp do drukarek sieciowych udostępnianych poprzez serwer wydruków. Serwer wydruków można zainstalować na oddzielnym serwerze, np. o nazwie \texttt{SPRXXfirma.ad}, lub w przypadku małych zasobów, na serwerze \texttt{SDC}. Dostęp do drukarki \texttt{\textbackslash\textbackslash SPRXX-firma.ad\textbackslash nazwa-drukarki}.
          \newpage
    \item Konfiguracja stacji klienckich w sposób automatyczny.
    \item Na klastrze pracy awaryjnej należy wdrożyć DHCP Serwer.
    \item Po udanym wdrożeniu serwera DHCP, wyłączyć serwer DHCP pracujący na SDC
    \item Wdrożyć serwer WWW wraz z firmową stroną (nie domyślną) dostępną pod adresem
          www.firma.ad.
    \item Wdrożyć WordPress współpracujący z usługą IIS Windows Serwera 2022.
\end{enumerate}
\subsection{Założenia projektowe}
\begin{itemize}
    \item nazwy serwerów powinny być zgodne ze schematem, przykładowo: SDCXX-NSACZ.firma.ad. - gdzie
          XX oznacza ostatnie dwie cyfry w numerze albumu,
    \item przyjąć adresację sieci jak niżej: 192.168.XX.40/24 - gdzie XX oznacza ostatnie dwie cyfry w numerze albumu,
    \item Zasoby sieciowe udostępnione na klastrze FC powinny być widoczne dla komputerów w postaci ścieżki UNC jako \texttt{\textbackslash\textbackslash sfsXX.firma.ad\textbackslash zasob}, \\ oraz \texttt{\textbackslash\textbackslash sfsXX.firma.ad\textbackslash wydzial} (nazwa wydziału), gdzie \texttt{XX} oznacza ostatnie dwie cyfry w numerze albumu.
    \item mapowanie dysków zgodnie ze schematem:
          zasób wspolny jako dysk X: , dyski wydziałowe(oddzielny zasób dla każdego wydziału) od G do K,
    \item  zaprojektować polisy tworzącą katalog c:\textbackslash ProjektyXX na stacjach końcowych, oraz ustawiającą
          zmienną środowiskową FIRMA=Nazwa, gdzie XX oznacza ostatnie dwie cyfry w numerze albumu,
    \item zakres adresacji IP dla stacji końcowych w zakresie od 100 ... do 200, uwzględnić funkcjonalność DHCP Serwer’a uruchomionego na klastrze,
    \item na stacji zainstalować drukarkę sieciową udostępnioną przez serwer wydruków i pokazać jej
          działanie.
\end{itemize}

   \newpage
\section{Opis użytych technologii}		%2
%(W podpunktach dokonać krótkiej charakterystyki użytych technologii ) 



   	\newpage
\section{Schemat logiczny}		%3
% (Ma zawierać aktualne nazewnictwo i adresację IP.)
   	\newpage
\section{Procedury instalacyjne poszczególnych usług}		%4
% Procedury instalacyjne poszczególnych usług.
% (W podpunktach zamieścić polecenia dotyczące instalacji wdrażanych usług) 

   	\newpage
\section{Testy działania wdrożonych usług}	%5
% (W podpunktach zamieścić zrzuty ekranów pokazujące działanie wdrożonych usług)


   	\newpage
\section{Wnioski}	%5
%Npisać wnioski końcowe z przeprowadzonego projektu, 



   
       
%%%%%%%%%%%%%%%%%%% koniec treść główna dokumentu %%%%%%%%%%%%%%%%%%%%%
	\newpage
    % \addcontentsline{toc}{section}{Literatura}
    % Modified by: Maciej Wójs  
    \printbibliography[heading=bibnumbered, label=Literatura, title=Literatura]

    \newpage
    \hypersetup{linkcolor=black}
    \renewcommand{\cftparskip}{3pt}
    \clearpage
    \renewcommand{\cftloftitlefont}{\Large\bfseries\sffamily}
    \listoffigures
    \addcontentsline{toc}{section}{Spis rysunków}
	\thispagestyle{fancy}
	
    \newpage
    \renewcommand{\cftlottitlefont}{\Large\bfseries\sffamily}
    \def\listtablename{Spis tabel}
    \addcontentsline{toc}{section}{Spis tabel}\listoftables 
	\thispagestyle{fancy}
	
	\newpage
	\renewcommand{\cftlottitlefont}{\Large\bfseries\sffamily}
	\renewcommand\lstlistlistingname{Spis listingów}
	\addcontentsline{toc}{section}{Spis listingów}\lstlistoflistings 
	\thispagestyle{fancy}
    \label{LastPage}
	


    %lista rzeczy do zrobienia: wypisuje na koñcu dokumentu, patrz: pakiet todo.sty
    \todos
    %koniec listy rzeczy do zrobienia
\end{document}
